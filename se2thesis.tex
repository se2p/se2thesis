\iffalse meta-comment

File: se2thesis.tex Copyright (C) 2022--2024 Stephan Lukasczyk

It may be distributed and/or modified under the conditions of the
LaTeX Project Public License (LPPL), either version 1.3c of this
license or (at your option) any later version.  The latest version
of this license is in the file

   https://www.latex-project.org/lppl.txt

This file is part of the "se2thesis bundle" (The Work in LPPL)
and all files in that bundle must be distributed together.

The released version of this bundle is available from CTAN.

---------------------------------------------------------------------

The development version of the bundle can be found at

   https://github.com/se2p/se2thesis

for those people who are interested.

---------------------------------------------------------------------

\fi

\documentclass{l3doc}

% The next line is needed so that \GetFileInfo will be able to pick up
% version data.
\usepackage{se2colors}

%
% Commands for this document, taken from Joseph Wright's siunitx
% documentation:
\ExplSyntaxOn
\makeatletter
\NewDocumentCommand \acro { m }
  {
    \textsc
      {
        \exp_args:NV \tl_if_head_eq_charcode:nNTF \f@series { m }
        { \text_lowercase:n }
        { \use:n }
          {#1}
      }
  }
\makeatother
\ExplSyntaxOff
\NewDocumentCommand{\email}{m}{\href{mailto:#1}{\nolinkurl{#1}}}
\NewDocumentCommand{\ext}{m}{\texttt.#1}
\NewDocumentCommand{\opt}{m}{\texttt{#1}}
% Tidy up the above in bookmarks
\makeatletter
\pdfstringdefDisableCommands{%
  \let\acro\@firstofone
  \let\ext\@firstofone
  \let\opt\@firstofone
}
\makeatother

% For creating code demonstration, taken from Joseph Wright's siunitx
% documentation:
\usepackage{listings}
\makeatletter
\lst@RequireAspects{writefile}
\newsavebox\LaTeXdemo@box
\lstnewenvironment{LaTeXdemo}[1][code and example]
  {%
    \global\let\lst@intname\@empty
    \edef\LaTeXdemo@end{%
      \expandafter\noexpand\csname LaTeXdemo@@#1@end\endcsname
    }%
    \@nameuse{LaTeXdemo@@#1}%
  }
  {\LaTeXdemo@end}
\newcommand\LaTeXdemo@new[3]{%
  \@namedef{LaTeXdemo@@#1}{#2}
  \@namedef{LaTeXdemo@@#1@end}{#3}%
}
\newcommand*\LaTeXdemo@common{%
  \setkeys{lst}
    {%
      basicstyle       = \small\ttfamily,
      breaklines       = true,
      basewidth        = 0.51em,
      captionpos       = t,
      extendedchars    = true,
      frame            = single,
      gobble           = 2,
      keywordstyle     = \color{blue}\bfseries,
      language         = [LaTeX]{TeX},
      showspaces       = false,
      showstringspaces = false,
      showtabs         = false,
      tabsize          = 2,
    }%
}
\newcount\LaTeXdemo@count
\newcommand*\LaTeXdemo@input{%
  \catcode`\^^M = 10\relax
  \input{\jobname-\number\LaTeXdemo@count.tmp}%
}
\LaTeXdemo@new{code and example}{%
  \setbox\LaTeXdemo@box=\hbox\bgroup
    \global\advance\LaTeXdemo@count by 1 %
    \lst@BeginAlsoWriteFile{\jobname-\number\LaTeXdemo@count.tmp}%
    \LaTeXdemo@common
}{%
    \lst@EndWriteFile
  \egroup
  \begin{center}
    \ifdim\wd\LaTeXdemo@box > 0.48\linewidth
      \begin{minipage}{\linewidth}
        \usebox\LaTeXdemo@box
      \end{minipage}%
      \par
      \begin{minipage}{\linewidth}
        \LaTeXdemo@input
      \end{minipage}
    \else
      \begin{minipage}{0.48\linewidth}
        \LaTeXdemo@input
      \end{minipage}%
      \hspace{\fill}%
      \begin{minipage}{0.48\linewidth}
        \usebox\LaTeXdemo@box
      \end{minipage}%
    \fi
  \end{center}
}
\LaTeXdemo@new{code and float}{%
  \global\advance\LaTeXdemo@count by 1 %
  \lst@BeginAlsoWriteFile{\jobname-\number\LaTeXdemo@count.tmp}%
  \LaTeXdemo@common
}{%
  \lst@EndWriteFile
  \LaTeXdemo@input
}
\LaTeXdemo@new{code only}{\LaTeXdemo@common}{}
\makeatother

\usepackage[UKenglish]{babel}
\usepackage{se2fonts}
\usepackage{hvlogos}

% Taken from xcolor.dtx
\makeatletter
\def\testclr#1#{\@testclr{#1}}
\def\@testclr#1#2{{\fboxsep\z@\fbox{\colorbox#1{#2}{\phantom{XX}}}}}
\makeatother

\usepackage{hvfloat}
\hypersetup{%
  allcolors=UPSE2-Blue,%
  pdftitle={se2thesis---A Thesis Class for the Chair of Software Engineering II
  at the University of Passau, Germany},%
  pdfauthor={Stephan Lukasczyk},
}
\usepackage[capitalise]{cleveref}

\begin{document}

\GetFileInfo{se2colors.sty}

\title{%
  \pkg{se2thesis}---A Thesis Class for the Chair of Software
  Engineering~II at the University of Passau, Germany%
  \thanks{This file describes \fileversion,
    last revised \filedate.}%
}

\author{%
  Stephan Lukasczyk%
  \thanks{%
    E-mail: \href{mailto:stephan@dante.de}{stephan@dante.de}%
  }%
}

\date{Released \filedate}

\pagenumbering{roman}

\maketitle

\begin{abstract}
  One can choose from a wide variety of templates to write a thesis.
  Many universities provide very rigorous style guides and force their
  students to obey to those guides, even though they might be questionable
  from a typographics point of view.
  Other universities do not provide such guides and leave it to their students
  to choose or come up with a template.
  The latter is causing very differently-looking theses.

  To avoid such a situation in the future this bundle combines several
  \LaTeX{} packages and classes for the use at the Chair of Software
  Engineering~II at the University of Passau.
  We provide, among others, a document class for theses that shall be
  used by our students.
  The bundle is designed in a way that one can use the basic components as
  standalone packages to allow their reuse for other projects.
\end{abstract}

\tableofcontents
\clearpage
\pagenumbering{arabic}

% Make the separate source files into a single whole, format-wise
\RenewDocumentCommand{\maketitle}{}{%
  \clearpage
}
\DeclareExpandableDocumentCommand{\thanks}{m}{}

\let\DelayPrintIndex\PrintIndex
\RenewDocumentCommand{\PrintIndex}{}{}

\DisableImplementation

\part{User Documentation}\label{sec:doc}

The first part of this file provides the documentation for the user
of the \pkg{se2thesis} bundle.
We provide the implementation in the second part of this document,
starting at page~\pageref{sec:impl}, for those who are curious.

% Load the source files in order
\DocInput{se2thesis.dtx}
\DocInput{se2colors.dtx}
\DocInput{se2fonts.dtx}
\DocInput{se2packages.dtx}

\EnableImplementation
\DisableDocumentation

\clearpage
\part{Implementation}\label{sec:impl}

The second part of this file provides the implementation of the package
for a better understanding of what is happening.

\DocInputAgain

\DelayPrintIndex

\end{document}